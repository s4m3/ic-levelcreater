\documentclass[10pt,a4paper]{article}

\usepackage[utf8]{inputenc}

\usepackage[T1]{fontenc}

\frenchspacing{}

\usepackage[english, ngerman]{babel}
\usepackage{textcomp}

\usepackage{hyperref}


%\usepackage{droid}

\usepackage[pdftex]{graphicx}

\usepackage[a4paper]{geometry}

\usepackage{setspace}
\setstretch{1.3}

%\usepackage{xhfill}
\usepackage{nth}
\usepackage[labelfont=bf,figurename=Fig.,tablename=Tab.]{caption}


%\definecolor{codeNote}{rgb}{0.0, 0.6, 0.0}
%\definecolor{gray-5}{gray}{0.85}
%\definecolor{orange}{rgb}{1.0, 0.27, 0}
%\definecolor{blue}{rgb}{0.0, 0.0, 0.8}

\usepackage{multicol}

\usepackage[activate]{pdfcprot}
\usepackage{listings}
\lstset{breaklines=true} 
%\lstset{numbers=left, numberstyle=\scriptsize\ttfamily, numbersep=10pt, captionpos=b} 
\lstset{columns=flexible}
\lstset{backgroundcolor=\color{gray-5}}
\lstset{basicstyle=\small\ttfamily}
\lstset{framesep=4pt}
\lstset{upquote=true}
\lstset{inputencoding=utf8}
\lstset{escapeinside={\%*}{*)}}
\lstset{language=bash}
\lstset{commentstyle=\color{blue},}
\lstset{
    literate={~} {$\sim$}{1}
}

\begin{document}

%%%%%%%%%%%%%   COMMANDS   %%%%%%%%%%%%%%%%%%%%%%%%%%%%%%%%%%%%%%%%%%%%%%%%%%%%%%%%
%\newcommand{\tip}{\rule{10pt}{10pt}\hspace{12pt}}

%\newenvironment{command}[1]{\noindent\xrfill{2pt}[orange]\textcolor{orange}{\textbf{#1}}\xrfill{2pt}[orange]}{\noindent\xrfill{2pt}[orange]}

%\newcommand{\codenote}[1]{\textcolor{codeNote}{#1}}
%%%%%%%%%%%%%   COMMANDS END   %%%%%%%%%%%%%%%%%%%%%%%%%%%%%%%%%%%%%%%%%%%%%%%%%%%%
	
	
\title{IC: Automatische Levelgenerierung}
\author{Simon Mary}
\maketitle
%\thispagestyle{empty}
%\newpage

%%%%%%%%%%%%%%%%%%%%%%%%%%%%%%%%%%%%%%%%%%%%%%%%%%%%%%%%%%%%%%%%%%%%%%%%%%%%%%%%%%%
\section{Einleitung}
Die hier beschriebene Arbeit wurde als \textit{Independent Production} im Rahmen eines \textit{Independent Coursework} an der \textit{HTW Berlin} bearbeitet. Dazu wurde eine automatisierte Level-Erstellung für eine 2D Simulation implementiert. Die Simulation ist eine Basisumgebung für intelligente Agenten in Form von Autos, die sich auf der Leveloberfläche bewegen. Diese Level beinhalten grundsätzlich fünf verschiedene Objekttypen, die auf der Karte als farbige Polygone oder Ellipsen dargestellt werden: 

%TODO: farben einfügen
\begin{enumerate}
\item{}Normaler Untergrund: Die Fahrzeuge können auf diesem Untergrund mit normaler Geschwindigkeit fahren.
\item{}Wände/Unbefahrbares Terrain: Die Fahrzeuge können nicht auf diesen Bereichen fahren.
\item{}Schwer befahrbarer Untergrund: Die Fahrzeuge können auf dieser Oberfläche mit verminderter Geschwindigkeit fahren.
\item{}Geschwindigkeitserhöhender Untergrund/Boost Terrain: Die Geschwindigkeit der Fahrzeuge wird auf diesem Untergrund erhöht.
\item{}Wegpunkte: Die Fahrzeuge müssen alle Wegpunkte abfahren, um das Level zu beenden.
\end{enumerate}

Dazu wird durch Verfahren der prozeduralen Levelgenerierung mithilfe einstellbarer Parameter das Terrain erstellt. Anders als bei einer völlig zufälligen Levelgenerierung, besitzen die Level eine Grundstruktur, die durch vorbestimmte Regeln (z.B. Wände am Rand) implementiert ist. Des weiteren werden die Level am Ende evaluiert (z.B. "sind alle Wegpunkte zugänglich"), sodass sie sinnvoll in die Simulation integriert werden können und nutzbar sind. Um eine weitere Nutzung der erstellten Level zu gewährleisten wurde außerdem eine Exportierfunktion zum Speichern des Levels integriert.

Die Umsetzung erfolgte in Java mithilfe von AWT und Swing zur Darstellung der Ergebnisse.

\section{Umsetzung}
%TODO: Alle Algorithmen aufzählen und kurz erklären
\subsection{Zellularer Automat}
%http://www.roguebasin.com/index.php?title=Cellular_Automata_Method_for_Generating_Random_Cave-Like_Levels
Um die Grundform des Levels zu erstellen wurde zunächst ein evolutionärer Algorithmus benutzt. Die Recherche nach einer geeigneten Methode führte schließlich zur Implementierung eines zellulären Automaten. %TODO: Quelle

Die Größe des Levels wird über die Parameter Breite und Höhe bestimmt. Ein zweidimensionales Array, initiiert aus Integern mit dem Wert 0 wird zunächst an zufälligen Stellen durch einen prozentualen Anteil n mit dem Wert 1 gefüllt. Die Stellen im Array, die mit 1 markiert sind, werden in einem späteren Schritt wie eine Wand behandelt, die Nullen sind später frei befahrbare Fläche. Da das Level grundsätzlich eine Außenwand haben soll, werden diese Punkte automatisch mit einer 1 gefüllt. Danach besteht das dadurch entstandene Level-Raster aus $n$ Prozent Einsen und $100-n$ Prozent Nullen. %TODO BILD. 
Dies stellt die Generation 0 des evolutionären Algorithmus dar.

Nun folgen mehrere Evolutionen oder auch Iterationen über das Raster -- iteriert wird über jeden Punkt im Raster -- mit folgenden einfachen Regeln des zellulären Automaten:
\begin{itemize}
\item{} Liegt der Punkt am Rand des Levels, bleibt eine 1 bestehen.
\item{} Ist der aktuelle Punkt eine 1 und
	\begin{itemize}
	\item{} hat vier oder mehr Nachbarn mit dem Wert 1, bleibt der Wert 1 bestehen.
	\item{} hat weniger als 2 Nachbarn mit dem Wert 1, wird der Wert auf 0 gesetzt.%TODO: sinnvoll? warum nicht weniger als 4?
	\end{itemize}
\item{} Ist der aktuelle Punkt eine 0 und hat fünf oder mehr Nachbarn mit dem Wert 1, wird der Wert auf 1 gesetzt.
\item{} In allen sonstigen Fällen wird der Wert auf 0 gesetzt.
\end{itemize}

Durch mehrere Iterationen dieses Algorithmus entstehen mehrere zusammenhängende Flächen, die später in Polygone umgewandelt werden können und als Wände fungieren. Die genutzten Parameter (prozentualer Anteil der gefüllten Fläche bei der Initialisierung, sowie die Regeln des zellulären Automaten) haben sich durch Testdurchläufe ergeben. Dazu ist anzumerken, dass geringe Veränderungen dieser Parameter eine große Auswirkung auf das Endresultat haben. %TODO: Verweis auf die Bilder. 

%TODO: bilder zeigen mit evolutionen


\subsection{Zusammenfassung der Punktmengen}
Um Polygone aus dem Grid... zunächst indexieren
Da aus den entstandenen Punktmengen Polygone (Level-Wände) entstehen sollen, müssen zunächst die jeweils zusammenhängenden Punkte als zusammengehörig deklariert werden. Dazu bekommt jeden Punktmenge im Raster ein eigenes Label. Um diesen Schritt möglichst performant durchzuführen wurde der \textit{Floodfill}-Algorithmus angewendet. %TODO: Quelle, Anschaulich: http://de.wikipedia.org/wiki/Floodfill




\subsection{Kontur tracing / Potrace/Make Polygons (convex und konkav) aus punktmenge}
Burger Burge etc.

\subsection{PolygonPointReducer um polygon zahl zu reduzieren}

\subsection{Konvex Polygone aus Punktmenge für slow down felder}

\subsection{speed up?}

\subsection{A stern für wegpunkte}

\subsection{Skalierung...}


\section{Programm/Implementierung}
%Programmstruktur etc.

\section{Ergebnisse}
%Beispiel Level

\section{Fazit}

%%%%%%%%%%%%%%%%%%%%%%%%%%%%%%%%%%%%%%%%%%%%%%%%%%%%%%%%%%%%%%%%%%%%%%%%%%%%%%%%%%%
\end{document}

