\documentclass[10pt,a4paper]{article}

\usepackage[utf8]{inputenc}

\usepackage[T1]{fontenc}

\frenchspacing{}

\usepackage[english, ngerman]{babel}
\usepackage{textcomp}

\usepackage{hyperref}


%\usepackage{droid}

\usepackage[pdftex]{graphicx}

\usepackage[a4paper]{geometry}

\usepackage{setspace}
\setstretch{1.3}

%\usepackage{xhfill}
\usepackage{nth}
\usepackage[labelfont=bf,figurename=Fig.,tablename=Tab.]{caption}


%\definecolor{codeNote}{rgb}{0.0, 0.6, 0.0}
%\definecolor{gray-5}{gray}{0.85}
%\definecolor{orange}{rgb}{1.0, 0.27, 0}
%\definecolor{blue}{rgb}{0.0, 0.0, 0.8}

\usepackage{multicol}

\usepackage[activate]{pdfcprot}
\usepackage{listings}
\lstset{breaklines=true} 
%\lstset{numbers=left, numberstyle=\scriptsize\ttfamily, numbersep=10pt, captionpos=b} 
\lstset{columns=flexible}
\lstset{backgroundcolor=\color{gray-5}}
\lstset{basicstyle=\small\ttfamily}
\lstset{framesep=4pt}
\lstset{upquote=true}
\lstset{inputencoding=utf8}
\lstset{escapeinside={\%*}{*)}}
\lstset{language=bash}
\lstset{commentstyle=\color{blue},}
\lstset{
    literate={~} {$\sim$}{1}
}

\begin{document}

%%%%%%%%%%%%%   COMMANDS   %%%%%%%%%%%%%%%%%%%%%%%%%%%%%%%%%%%%%%%%%%%%%%%%%%%%%%%%
%\newcommand{\tip}{\rule{10pt}{10pt}\hspace{12pt}}

%\newenvironment{command}[1]{\noindent\xrfill{2pt}[orange]\textcolor{orange}{\textbf{#1}}\xrfill{2pt}[orange]}{\noindent\xrfill{2pt}[orange]}

%\newcommand{\codenote}[1]{\textcolor{codeNote}{#1}}
%%%%%%%%%%%%%   COMMANDS END   %%%%%%%%%%%%%%%%%%%%%%%%%%%%%%%%%%%%%%%%%%%%%%%%%%%%
	
	
\title{IC: Automatische Levelgenerierung}
\author{Simon Mary}
\maketitle
%\thispagestyle{empty}
%\newpage

%%%%%%%%%%%%%%%%%%%%%%%%%%%%%%%%%%%%%%%%%%%%%%%%%%%%%%%%%%%%%%%%%%%%%%%%%%%%%%%%%%%
\section*{Abstract}
Die hier beschriebene Arbeit wird als \textit{Independent Production} im Rahmen eines \textit{Independent Coursework} an der \textit{HTW Berlin} bearbeitet. Dazu wird eine automatisierte Level-Erstellung für eine 2D Simulation implementiert. Die Simulation ist eine Basisumgebung für intelligente Agenten in Form von Autos, die sich auf der Leveloberfläche bewegen werden. Diese Level beinhalten grundsätzlich fünf verschiedene Objekttypen, die auf der Karte als farbige Polygone oder Ellipsen dargestellt werden: 

\begin{enumerate}
\item{}Normaler Untergrund: Die Fahrzeuge können auf diesem Untergrund mit normaler Geschwindigkeit fahren.
\item{}Wände/Unbefahrbares Terrain: Die Fahrzeuge können nicht auf diesen Bereichen fahren.
\item{}Schwer befahrbarer Untergrund: Die Fahrzeuge können auf dieser Oberfläche mit verminderter Geschwindigkeit fahren.
\item{}Geschwindigkeitserhöhender Untergrund/Boost Terrain: Die Geschwindigkeit der Fahrzeuge wird auf diesem Untergrund erhöht.
\item{}Wegpunkte: Die Fahrzeuge müssen alle Wegpunkte abfahren, um das Level zu beenden.
\end{enumerate}

Dabei soll durch Verfahren der prozeduralen Levelgenerierung mithilfe einstellbarer Parameter das Terrain erstellt werden. Anders als bei einer völlig zufälligen Levelgenerierung, werden die Level eine Grundstruktur aufweisen (z.B. Wände am Rand) und am Ende evaluiert (z.B. sind alle Wegpunkte zugänglich?) werden, sodass sie sinnvoll in die Simulation integriert werden können und nutzbar sind. Um eine weitere Nutzung der erstellten Level zu gewährleisten wird außerdem eine Exportierfunktion zum Speichern des Levels in ein gängiges Format (z.B. XML) integriert.
Die Umsetzung erfolgt in Java mithilfe von AWT und Swing zur Darstellung der Ergebnisse.


%%%%%%%%%%%%%%%%%%%%%%%%%%%%%%%%%%%%%%%%%%%%%%%%%%%%%%%%%%%%%%%%%%%%%%%%%%%%%%%%%%%
\end{document}

